%!TEX root = structure.tex

\section{Définition du prototype \textit{System}}
Qu'est-ce qui définit un système dynamique comme je les conçois? Essayons un premier jet.

\begin{lstlisting}
var System = function(system){
    this.pos = system.pos || {x: 0, y: 0, z: 0};
    this.n = system.n || 0;
    this.animation = system.animation || false;
    this.backgroundColor = system.backgroundColor || 0;
    this.setup = system.setup;
    this.iterativeFunction = system.iterativeFunction;
    this.displayFunction = system.displayFunction;
    this.animation = system.animation || false;
};

System.prototype.runIterativeFunction = function() {
    this.pos = this.iterativeFunction(this.pos.x, this.pos.y, this.pos.z, this.constants);
    this.n++;
};

System.prototype.runDisplayFunction = function() {
    this.displayFunction(this.pos, this.n, this.translate, this.scale);
};
\end{lstlisting}

\newpage
\section{Définition des systèmes individuels}

\begin{lstlisting}
var dune001 = new System({
    pos: {x: 0, y: 0, z: 0},
    n: 0,
    constants: {
        a: 5,
        b: 3,
        c: 2,
        d: 1,
        e: 2,
        f: 1
    },
    animation: false,
    iterationsPerFrame: 1500,
    backgroundColor: 0,
    setup: function() {
        frameRate(30);
        background(0);
        fill(255, 5);
        noStroke();
    },
    iterativeFunction: function(x, y, z, c) {
        //c is a set of constants.
        var v = {
            x: sin(c.a * x) + cos(c.b * y) - tan(c.c * z),
            y: cos(c.d * y) + cos(c.e * x) + tan(c.f * z),
            z: z + 0.1
        };
        return v;
    },
    displayFunction: function(v, n) {
        var red = map(abs(sin(n / 10)), 0, 1, 255, 100);
        var green = map(abs(cos(n / 30)), 0, 1, 200, 40);
        var blue = map(abs(sin(n / 10)), 0, 1, 0, 255);
        blendMode(ADD);
        fill(red, green, blue, 15);
        ellipse(v.x * 100, v.y * 100, 0.5);
    }
});
\end{lstlisting}

Les systèmes devraient-ils également avoir des valeurs \textit{scale} et \textit{translate}? Ça me permettrait de réutiliser les mêmes fonctions d'affichage mais de modifier la zone affichée. D'expérience, c'est quelque chose que je voudrai faire très souvent.

\newpage
\section{Définition de fonctions indépendantes}
Les fonctions itératives peuvent être définies indépendamment des systèmes dynamiques, et utilisées par ceux-ci. Même chose pour les fonctions d'affichage.
\begin{lstlisting}
var duneFunction = function(x, y, z, c) {
    var v = {
        x: sin(c.a * x) + cos(c.b * y) - tan(c.c * z),
        y: cos(c.d * y) + cos(c.e * x) + tan(c.f * z),
        z: z + 0.1
    };
    return v;
};

var colorFunction001 = function(v, n, translate, scale) {
    var red = map(abs(sin(n / 10)), 0, 1, 255, 100);
    var green = map(abs(cos(n / 30)), 0, 1, 200, 40);
    var blue = map(abs(sin(n / 10)), 0, 1, 0, 255);
    blendMode(ADD);
    fill(red, green, blue, 15);
    ellipse(translate.x + v.x * scale.x, translate.y + v.y * scale.y, 0.5);
};
\end{lstlisting}
\newpage
\section{Une possible fonction pour dupliquer un système}
Devrais-je me créer une fonction qui ne servirait qu'à dupliquer un système entier, pour ensuite le modifier? Ça pourrait ressembler à ça :

\begin{lstlisting}
var dune003 = dune002.clone();
dune003.scale = {x: 200, y: 0};
\end{lstlisting}

Et la fonction copiante ressemblerait à ça :
\begin{lstlisting}
System.prototype.clone = function() {
    return {
        pos: {x: this.pos.x, 
            y: this.pos.y, 
            z: this.pos.z},
        n: this.n,
        constants: this.constants,
        animation: this.animation,
        iterationsPerFrame: this.iterationsPerFrame,
        backgroundColor: this.backgroundColor,
        setup: this.setup,
        iterativeFunction: this.iterativeFunction,
        displayFunction: this.displayFunction
    };
};
\end{lstlisting}

En fait, peut-être qu'un système pourrait tout simplement cloné (et modifié) comme ça :
\begin{lstlisting}
var dune003 = new System(dune002);
dune003.scale = {x: 200, y: 0};
\end{lstlisting}